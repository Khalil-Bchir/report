\section{Work Environment}

This section outlines the work environment for the project, including the hardware environment, development environment, and software environment.

\subsection{Software Environment}

In our project, we used the following tools:

\subsubsection{Visual Studio Code}
\begin{center}
\includegraphics[width=0.15\textwidth]{Images/logos/vscode.png}
\captionof{figure}{Visual Studio Code logo}
\label{fig:vscode}
\end{center}
The development environment for this project is a combination of the software tools mentioned in the Software Environment section. Each tool plays a crucial role in different aspects of the development process.

\subsubsection{GitLab}
\begin{center}
\includegraphics[width=0.2\textwidth]{Images/logos/gitlab-logo-500.png}
\captionof{figure}{GitLab logo}
\label{fig:gitlab}
\end{center}
GitLab serves as the remote repository for code hosting and versioning. It also provides a platform for creating CI/CD pipelines, facilitating continuous integration and deployment of the project.

\subsubsection{Notion}
\begin{center}
\includegraphics[width=0.2\textwidth]{Images/logos/62cc159e150d5de9a3dad5ec.png}
\captionof{figure}{Notion logo}
\label{fig:notion}
\end{center}
Notion serves as the central hub for project management. It is used for task tracking, documentation, and collaboration, providing a unified workspace for all project-related information and tasks.

\subsubsection{Postman}
\begin{center}
\includegraphics[width=0.2\textwidth]{Images/logos/62cc1b6b150d5de9a3dad5f9.png}
\captionof{figure}{Postman logo}
\label{fig:postman}
\end{center}
Postman is used for testing and debugging APIs. It simplifies the process of building, testing, and modifying APIs, making it an essential tool for backend development.

\subsubsection{DBeaver}
\begin{center}
\includegraphics[width=0.2\textwidth]{Images/logos/DBeaver_logo.png}
\captionof{figure}{DBeaver logo}
\label{fig:dbeaver}
\end{center}
DBeaver is used for database management and interaction. It supports a wide range of databases and provides a user-friendly interface for managing database schemas, running SQL queries, and performing other database-related tasks.

\subsubsection{Windows Subsystem for Linux}
\begin{center}
\includegraphics[width=0.2\textwidth]{Images/logos/Windows_Subsystem_for_Linux_logo.png}
\captionof{figure}{Windows Subsystem for Linux logo}
\label{fig:wsl}
\end{center}
The Windows Subsystem for Linux (WSL) allows developers to run a Linux environment directly on Windows, without the overhead of a traditional virtual machine or dual-boot setup. This makes it possible to use Linux command-line tools and utilities alongside Windows applications.

\subsubsection{Docker Desktop}
\begin{center}
\includegraphics[width=0.2\textwidth]{Images/logos/docker-mark-blue.png}
\captionof{figure}{Docker Desktop logo}
\label{fig:docker}
\end{center}
Docker Desktop is used to containerize the services. It allows for the packaging and distribution of applications in a manner that is independent of the host operating system, ensuring consistent operation across different computing environments.

\subsubsection{Git}
\begin{center}
\includegraphics[width=0.2\textwidth]{Images/logos/Git-Icon-1788C.png}
\captionof{figure}{Git logo}
\label{fig:git}
\end{center}
Git is used for version control, allowing for efficient tracking and management of changes to the project codebase. It is an essential tool for collaborative development.
\subsection{Development Environment}

\subsubsection{Node.js}

\begin{center}
\includegraphics[width=0.15\textwidth]{Images/logos/node.png}
\captionof{figure}{Node.js logo}
\label{fig:nodejs}
\end{center}
Node.js is a JavaScript runtime built on Chrome’s V8 JavaScript engine. It’s used for building scalable network applications and executing JavaScript code server-side.



\subsubsection{TypeScript}

\begin{center}
\includegraphics[width=0.15\textwidth]{Images/logos/ts-lettermark-blue.png}
\captionof{figure}{TypeScript logo}
\label{fig:typescript}
\end{center}
TypeScript is a typed superset of JavaScript that adds static types. It helps to write more robust code and maintain a cleaner and more consistent codebase. It’s used in this project along with Vue 3 for building user interfaces.

\subsubsection{Vue 3}

\begin{center}
\includegraphics[width=0.15\textwidth]{Images/logos/vue.png}
\captionof{figure}{Vue 3 logo}
\label{fig:vue3}
\end{center}
Vue 3 is the latest version of Vue.js, a progressive JavaScript framework for building user interfaces. TypeScript is a typed superset of JavaScript that adds static types. It helps to write more robust code and maintain a cleaner and more consistent codebase.

\subsubsection{HTML}

\begin{center}
\includegraphics[width=0.15\textwidth]{Images/logos/html.png}
\captionof{figure}{HTML logo}
\label{fig:html}
\end{center}
HTML (HyperText Markup Language) is the standard markup language for documents designed to be displayed in a web browser. It’s used in this project for structuring the content on the webpages.

\subsubsection{Tailwind CSS}

\begin{center}
\includegraphics[width=0.15\textwidth]{Images/logos/tailwind.png}
\captionof{figure}{Tailwind CSS logo}
\label{fig:tailwind}
\end{center}
Tailwind CSS is a utility-first CSS framework for rapidly building custom user interfaces. It provides low-level utility classes that let you build completely custom designs without ever leaving your HTML.

\subsubsection{Fastify}

\begin{center}
\includegraphics[width=0.15\textwidth]{Images/logos/fastify.png}
\captionof{figure}{Fastify logo}
\label{fig:fastify}
\end{center}
Fastify is a web framework highly focused on providing the best developer experience with the least overhead and a powerful plugin architecture. It is used for building efficient and scalable Node.js web applications.

\subsubsection{MJML}

\begin{center}
\includegraphics[width=0.15\textwidth]{Images/logos/file-type-mjml.512x453.png}
\captionof{figure}{MJML logo}
\label{fig:MJML}
\end{center}
MJML is a markup language designed to simplify the process of coding responsive emails. It provides a semantic syntax and a library of standard components, making it easier to build custom email templates that are responsive and compatible with various email clients.

\subsubsection{Turborepo}

\begin{center}
\includegraphics[width=0.15\textwidth]{Images/logos/turbopack-logotype-light-background.png}
\captionof{figure}{Turborepo logo}
\label{fig:turborepo}
\end{center}
Turborepo is a high-performance build system for JavaScript and TypeScript codebases. It provides features like incremental builds, parallel execution, and remote caching. It’s designed for monorepos, improving build times and facilitating efficient project management.

\subsubsection{Prisma}

\begin{center}
\includegraphics[width=0.15\textwidth]{Images/logos/Prisma.png}
\captionof{figure}{Prisma logo}
\label{fig:prisma}
\end{center}
Prisma is an open-source database toolkit. It replaces traditional ORMs and makes database access easy with an auto-generated and type-safe query builder that’s tailored to your database schema.


\subsubsection{Swagger}

\begin{center}
\includegraphics[width=0.15\textwidth]{Images/logos/Swagger-logo.png}
\captionof{figure}{Swagger logo}
\label{fig:swagger}
\end{center}
Swagger is an open-source software framework backed by a large ecosystem of tools that helps developers design, build, document, and consume RESTful web services. It’s used in this project for API design and documentation.
